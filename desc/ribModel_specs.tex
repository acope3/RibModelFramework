
\documentclass[letter,10pt]{article}
\usepackage[utf8]{inputenc}

%opening
\title{ribModel Code specifications}


\begin{document}
\maketitle


\begin{enumerate}
 \item new function should contain a description above the implementation that state purpose of function and explains parameter. Note if function is exposed to R or not.
 \item create a test case for new functions in class Testing.
 \item use speaking variable names
 \item if you copy code, mention source so code can be double checked later.
 \item ensure that constructors create a VALID object. That does not necessary mean all information has to be placed, but all members have to be properly initialized. 
 \item document/describe parts of code/functions that are not self-explenatory (Can you come back to it a month later and still know what the code does).
 \item functions are only defined in headers, but never implemented.
 \item avoid invinite loop structures ( for(;;), while(true), ... ) that break under various conditions.
 \item new classes should follow naming convention [MODEL]Parameter, [MODEL]Model. That should be generally applied for inhereted classes.
 \item avoid dynamic allocated arrays and use vectors instead if possible. dynamic allocated arrays seem to cause problems with openmp.
 \item all log output should use the wrapper functions ``my\_print'', ``my\_printError'', ``my\_printWarning''
 \item Rcpp modules are used to expose classes and member functions. Member functions and static functions are exposed using a different syntax (see code). 
 \item Constructors can not have more than 6 arguments. That causes us to use setter functions in cases where more arguments are needed. Adjustment can be made once Rcpp can handle more arguments.
 \item stick to the google style guide for R
\end{enumerate}


\end{document}

